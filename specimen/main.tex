%!xelatex
\documentclass{memoir}
\setstocksize{594mm}{420mm}\settrimmedsize{594mm}{420mm}{*}
\setlrmarginsandblock{1in}{1in}{*}
\newlength{\FootSkip}\setlength{\FootSkip}{9.6pt}
\addtolength{\FootSkip}{\onelineskip}
\newlength{\LowerMargin}\setlength{\LowerMargin}{1in}
\addtolength{\LowerMargin}{\FootSkip}
\setulmarginsandblock{.99in}{\LowerMargin}{*}
\setheadfoot{12pt}{\FootSkip}\setheaderspaces{*}{0in}{*}
\setmarginnotes{10pt}{30pt}{10pt}
\checkandfixthelayout
\usepackage{bigstrut}
\usepackage[no-math,quiet]{fontspec}
\defaultfontfeatures{Ligatures={Required,Common,Contextual,Rare},CharacterVariant={6}}
\setmainfont{EB Garamond}[ItalicFeatures={RawFeature=+ligh}]
\newfontfamily{\morespace}{EB Garamond}[WordSpace={1.5}]
\frenchspacing
\usepackage{polyglossia}\setdefaultlanguage{dutch}
\usepackage{multicol}
\usepackage{hyperref}
\hypersetup{%
    pdftitle={De Raaf},
    pdfsubject={Poetry / American / General},
    pdfauthor={Edgar Allan Poe vertaalt door Gerard Den Brabander in 1943},
    pdfkeywords={de raaf, raaf, poëzie, edgar allan poe, poe}
}
\pagestyle{empty}
\begin{document}

% http://resources21.kb.nl/gvn/EVDO03/pdf/EVDO03_UBL01_DEJONG_1002.pdf

\begin{multicols}{2}
\huge\noindent\strut\\\strut\\\strut\\\strut\\{\centering {\strut\hspace{-16mm}\HUGE{\morespace\addfontfeature{LetterSpace=15} DE RAAF}}\\
\strut\\
\hspace{-16mm}{\addfontfeature{RawFeature={+smcp}}Naar}\\\strut\\
\hspace{-16mm}EDGAR ALLAN POE, {\addfontfeature{Numbers=Lining}1849}\\
\hspace{-16mm}{\addfontfeature{RawFeature={+smcp}}Vertaald door} GERARD DEN BRABANDER {\addfontfeature{RawFeature={+smcp}}in} {\addfontfeature{Numbers=Lining}1943}\\}\strut\\\strut\\\strut\\\vspace{-2mm}

\noindent Op een nacht, toen 'k mat en moede peinsde, piekerde en broedde\\
op het vele en onvermoede van een leer die ging teloor;\\
toen ik doezelig zat te knikken, deed er iets mij lichtelijk schrikken;\\
drong er in mijn droomerig nikken zacht een tikken tot mij door.\\
„Zou dat een bezoeker wezen?” mompelde ik mijzelven voor.\\
\indent\indent „'t Kan niet anders, komt mij voor.”\\

\noindent Ik herinner mij die uren duidelijk; dien December guur en\\
op het vloerkleed de figuren, stervend in den sintelgloor.\\
Vurig beidde ik den morgen, want geen boekdeel wilde borgen\\
de bevrijding uit mijn zorgen: zorg en smart om Eleonoor;\\
zorg en smart om 't stralend wezen bij dë engelen, Eleonoor;\\
\indent\indent naamloos {\itshape hier}, àl de eeuwen door.\\

\noindent Somber, zijden, onbestendig rilde en ritselde 't lamlendig\\
purperen gordijn: ellendig huiverde het mijn leden door,\\
dat ik, tegen het gehamer van mijn hart, zei tot de kamer:\\
„Een bezoeker, nòg eenzamer, dringt tot mijn alleen-zijn door;\\
een bezoeker, die verlaat is, dringt tot mijn alleen-zijn door.\\
\indent\indent Dat is alles wat ik hoor.”\\

\noindent En mijn ziel werd van halfslachtig aarzelend weer kalm en krachtig.\\
„Heer,” zei ik, „of dame, waarlijk, schenk mij daar vergiffenis voor,\\
want ik doezelde en ik knikte, toen gij zóó zachtzinnig tikte,\\
dat mijn droombeeld nauwlijks schrikte, toen gij tikte om gehoor.\\
Zie mijn deur, zij zwenkt naar binnen, Laat U zien. Ik sta ervoor.”\\
\indent\indent Duisternis als nooit tevoor.\\

\noindent Diep in 't duister voor mij glurend, stond ik daar verwonderd, turend,\\
twijfelend, durvend, droomen droomend als geen sterveling tevoor.\\
Maar de stilte wou niet breken en het duister zond geen teeken\\
dan alleen 't eenzelvig spreken van mijn fluisterend „Eleonoor.”\\
En dë echo bracht mij murmelend slechts dit eene „Eleonoor.”\\
\indent\indent Enkel dit kwam tot mij door.\\

\noindent In de kamer wederkeerend; innerlijk door vuur verterend,\\
hoorde ik weldra weer het tikken, ietwat luider dan tevoor.\\
„Of 't een Wie, dan wel een Wat is, niets is duidelijker dan dat is:\\
in het duister, dat een gat is, kom ik dit geheim op 't spoor;\\
als mijn hartslag kalm en mat is kom ik dit geheim op 't spoor:\\
\indent\indent 't is de wind slechts, die ik hoor.”\\

\noindent Bliksemsnel een luik ontwerveld! Door den wind erin gewerweld,\\
stapte statig daar een raaf uit lang vervlogen eeuwen door.\\
Zelfs de lichtste buiging meed-ie; stom en zonder stilstaan schreed-ie;\\
met het air van lord of lady stapte hij de kamer door;\\
stapte hij naar 't beeld van Pallas en zat statig op diens oor;\\
\indent\indent zat en zweeg op Pallas' oor.\\

\noindent 't Ebben vogelbeest verlokte, waar 't zoo stram op Pallas stokte,\\
mijn verdriet door zijn decorum: aarzelend brak mijn glimlach door.\\
„Kort van kuif en zwart van verve, zult gij, dunkt mij, moed niet derven,\\
raaf of spookbeeld, bij uw zwerven over 't nachtelijk spoor.\\
Zeg mij, lordschap, hoe uw naam luidt daar op Pluto's nachtelijk spoor.”\\
\indent\indent „Nooit meer”, kraste het in mijn oor.\\

\noindent Wonderlijk de redenatie van dien vogel zonder gratie,\\
woorden sprekend zonder rede, die hij redeloos verloor\\
Daarbij kwam de vraag gerezen: Was, als ik, één sterfelijk wezen\\
ooit gezegend als met dezen vogel daar op Pallas' oor;\\
met zoo'n beest als op het bortbeeld daar omhoog; op Pallas' oor?\\
\indent\indent „Nooit meer”, heette het, kwam mij voor.\\

\noindent Maar het beest, hoog en alleenig daar op 't borstbeeld, sprak slechts 't eenig'\\
eenigë waaraan hij blijkbaar ziel en zaligheid verloor.\\
Verder star en stil gezwegen, — vlerk nog staartveer zag 'k bewegen —\\
zat ik mopperend te overwegen hoeveel vrienden ik al verloor:\\
„In den morgen vlucht en vliedt {\itshape hij}, als de droomen, die 'k verloor.”\\
\indent\indent „Nooit meer”, kwam de raaf mij voor.\\

\noindent Bevend — door de bruusk gesproken woorden werd de sfeer verbroken —\\
dacht ik: „Alles wat hij in heeft is het antwoord, dat ik hoor.\\
Vastgekoppeld aan een wezen, dat de klauw van ramp en vreezen\\
rond de keel voelt en 't verwezen lied tot dit refrein bevroor;\\
tot de lijkzang op zijn droombeeld tot dit triest refrein bevroor;\\
\indent\indent tot dit „nooit meer, nooit” bevroor.\\

\noindent Maar, terwijl de raaf daar stokte en tot een glimlach mij verlokte,\\
sleepte ik mijn liefsten zetel voor de buste en 't beest op 't oor.\\
In dien zetel zat en zonk ik; in mijn diepst gepeins verdronk ik;\\
droom aan duister droombeeld klonk ik. Wat had deze vogel voor;\\
wat had deze barre vogel met zijn barsche „Nooit meer”, voor,\\
\indent\indent dat daar kraste aan mijn gehoor?\\

\noindent Onder al die raadselen zwichtend; mij met geen sylabe richtend\\
tot het beest, welks blikken brandend drongen in mijn boezem door,\\
dacht en peinsde ik. Ondertusschen liet het dwaze brein zich sussen\\
door het violette kussen onder lamplichts gloed en gloor.\\
Ach, wèlk hoofd er ooit mocht rusten op 't fluweel in lamplichts gloor,\\
\indent\indent nooit meer dat van {\itshape Eleonoor}!\\

\noindent Dacht mij toen, de lucht werd tastbaar, of de hemel zelf tegast waar';\\
reuk en rijkdom; alles was daar: ruischend, rinkelend engelspoor!\\
„Schelm!” riep ik, „God zelf bekent je en dë engelenschaar verwent je,\\
— zwelg en drink! — want deze zend je rust, respijt om Eleonoor!\\
Zwelg en drink dus uit dien beker en vergeet Eleonoor!”\\
\indent\indent „Nooit meer”, kraste het in mijn oor.\\

\noindent „Stomme vogel, ondier, euvel; zwijgende profeet of duivel;\\
wie of wat, je bent uit Satan of een stormbewogen spoor;\\
eenzaam steeds, zonder versagen; scheppende in dit oord behagen;\\
in dit huis van spook en plagen, zeg 't mij, die dit land verkoor,\\
{\itshape is} er troost nog? zeg 't mij, smeek ik; zeg 't mij, die dit land verkoor!”\\
\indent\indent „Nooit meer”, kraste het in mijn oor.\\

\noindent „Stomme vogel, ondier, euvel; zwijgende profeet of duivel!\\
Bij de hemelen daarboven; bij den God, dien ik behoor,\\
zeg mij — ik heb zwaar geleden — zit er in 't onvindbaar Eden\\
niet een zalige en aanbeden maagd gevangen: Eleonoor?\\
Zit geen ijl en stralend wezen daar in 't Eden: Eleonoor?”\\
\indent\indent „Nooit meer,” kraste het in mijn oor.\\

\noindent „Laat mijn woede niet ontsteken! Zij dit woord ons afscheidsteeken!\\
Ding of demon, grijp den storm weer op je nachtelijke spoor!\\
Laat geen veer hier van je spreken, want je ziel gaf taal noch teeken;\\
laat mij eenzaam, neergestreken duivel daar op Pallas' oor!”\\
Neem je bek weer uit mijn boezem en je beeld van Pallas' oor.\\
\indent\indent„Nooit meer”, kraste het in mijn oor.\\

\noindent En de raaf bewoog niet éven; zat daar stil en {\itshape zonder} leven\\
het witte, witte borstbeeld daar omhoog; op Pallas' oor;\\
en zijn blik was opgenomen in een duivelachtig droomen;\\
en het licht goot met zijn stroomen óók diens schaduw ver naar voor;\\
mijn ziel zal van dien schaduw nooit verlost zijn, de eeuwen door;\\
\indent\indent nooit verlost zijn, de eeuwen door!\\
\noindent{\normalsize 't Spuigat, Amsterdam, Nederland, 1943, p. 1–6. Opgemaakt in EB Garamond door S. van Geloven.}

\end{multicols}

\end{document}
